\newpage
\chapter{Vektorrum og underrum}
([P] 2.1, 3.2)

\section*{Disposition}
\begin{enumerate}
	\item Vektor- og Underrum
	\item Span og Basis
	\item Entydig Uafhængighed
	\item Underrum - Matrix
\end{enumerate}

\section{Vektor- og Underrum}
\begin{definition}{2.1.1}
Et vektorrum er en mængde $V$ udstyret med addition og skalarmultiplikation
således at følgende egenskaber er overholdt:

	\begin{description}
		\item[A1] $\forall x,y \in V \colon x+y=y+x$
		\item[A2] $\forall x,y,z \in V \colon (x+y)+z=x+(y+z)$
		\item[A3] $\exists 0 \in V \colon x+0=x \, \forall x \in V$
		\item[A4] $\forall x \in V, \, \exists -x \in V \colon x+(-x)=0$
		\item[S1] $\forall \alpha \in \mathbb{F}, \, x,y \in V \colon
			\alpha (x+y)=\alpha x+ \alpha y$
		\item[S2] $\forall \alpha, \beta \in \mathbb{F}, \, x \in V \colon
			(\alpha + \beta)x=\alpha x + \beta x$
		\item[S2] $\forall \alpha, \beta \in \mathbb{F}, \, x \in V \colon
			(\alpha\beta)x=\alpha (\beta x)$
		\item[S4] $\exists 1 \in \mathbb{F}, \, \forall x \in V \colon 1 \cdot x=x$
	\end{description}
\end{definition}


\begin{saetning}{2.1.2}
Et vektorrum har nogle elementære egenskaber
\begin{enumerate}
		\renewcommand{\theenumi}{\Roman{enumi}}
		\renewcommand{\labelenumi}{\textbf{(\theenumi)}}
	\item 0 er det entydige neutrale element i $V$
	\item $0\cdot x=0 \, \forall x \in V$
	\item Hvis $x,y\in V$ er således at $x+y=0$, så er $y=-x$
	\item $(-1)x = -x \, \forall x\in V$
\end{enumerate}
\end{saetning}


\hypertarget{def:underrum}{}
En delmængde $S$ af et $\mathbb{F}$-vektorrum $V$ kaldes et underrum, hvis de
har følgende egenskaber:
\begin{description}
	\item[C0] $S \not= \emptyset$
	\item[C1] $\forall x \in S, \, \alpha \in \mathbb{F} \colon \alpha x \in S$
	\item[C2] $\forall x,y \in S \colon x + y \in S$
\end{description}

\section{Span og Basis}
\begin{definition}{2.2.1}
	Lad $V$ være et $\mathbb{F}$-vektorrum og lad $v_1,\dots,v_n \in V$. 

	En vektor givet ved $\alpha_1 v_1+\dots+\alpha_n v_n \in V$ hvor
	$\alpha_1,\dots,\alpha_n \in \mathbb{F}$ er en lineær kombination af
	$v_1,\dots,v_n$. 
	
	Mængden af alle lineære kombinationer af $v_1,\dots,v_n$ kaldes spannet af
$v_1,\dots,v_n$, skrevet $\spn(v_1,\dots,v_n)$.
\end{definition}


$Span(v_1,\dots,v_n)$ er et underrum af $V$, idet spannet overholder C0 - C2.
Se eventuelt beviset side 34 [duPlessis].

\begin{definition}{2.2.11}
	Lad $V$ være et $\mathbb{F}$-vektorrum, og lad $v_1, \dots, v_n \in V$. Da
	er $v_1, \dots, v_n$ en basis for $V$ hvis:
	\begin{enumerate}
		\item $v_1, \dots, v_n$ er lineært uafhængige.
		\item $v_1, \dots, v_n$ udspænder (spanner) $V$.
	\end{enumerate}
\end{definition}


\section{Entydig Uafhængighed}
\begin{definition}{2.2.6}
	Et sæt $v_1, \dots, v_n$ af vektorer er lineært uafhængige såfremt der
	gælder at en lineær kombination af vektorerne 
	\[c_1v_1 + \dots + c_nv_n = 0\]
	kun har den trivielle løsning, altså at $c_i = 0$ for alle $i = 1,\dotsc,
	n$. Hvis der derimod eksisterer et $c_i \not= 0$, så samme ligning stadig
	er opfyldt, så er vektorerne istedet lineært afhængige, da en kan skrives
	som en lineær kombination af de andre.
\end{definition}


\begin{saetning}{2.2.9}
	Lad $V$ være et $\mathbb{F}$-vektorrum, lad $v_1, \dots, v_n \in V$, og lad
	$S = \spn(v_1, \dots, v_n)$. Et element $v \in S$ kan udtrykkes 
	\textit{entydigt} som en lineær kombination af $v_1, \dots, v_n \Lra v_1, 
	\dots, v_n$ er lineært uafhængige.
\end{saetning}

\begin{bevis}
Da $v \in S$ ved vi per definition af spannet, at vi kan skrive $v$ som en 
lineær kombination.

\[
	v = a_1 v_1 + \dots + a_n v_n \text{ med } a_1, \dots a_n \in \mathbb{F}
\]

Hvis der samtidig gælder at vi kan skrive v som:

\[
	v = b_1 v_1 + \dots + b_n v_n \text{ med } b_1, \dots b_n \in \mathbb{F}
\]

så er 

\[
	0 = v - v = (a_1-b_1)v_1 + \dots + (a_n-b_n)v_n
\]

Hvis $v_1, \dots, v_n$ er lineært \textbf{uafhængige} så må:

\[
	(a_1-b_1) = 0, \dots, (a_n-b_n) = 0 \Lra a_1 = b_1, \dots, a_n = b_n
\]

Der findes dermed kun 1 måde at udtrykke $v$ som en lineær kombination af 
$v_1, \dots, v_n$

Hvis $v_1, \dots, v_n$ er lineært \textbf{afhængige}, må der findes $c_1, 
\dots, c_n \in \mathbb{F}$, hvor ikke alle $c_i$ er 0, og hvor $c_1 v_1 + \dots
c_n v_n = 0$, hvilket giver os:

\[
	v = v + 0 = (a_1 v_1 + \dots + a_n v_n) + (c_1 v_1 + \dots c_n v_n)
	= (a_1 + c_1)v_1 + \dots + (a_n + c_n)v_n
\]

Altså er $(a_1 + c_1)v_1 + \dots + (a_n + c_n)v_n$ et andet udtryk for en
lineær kombination af $v$, da mindst 1 $c$ ikke er 0 og ligningen derfor ikke
er lig den øverste.
\end{bevis}


\section{Basis for Søjlerum}

\begin{saetning}{3.2.6}
	Lad $A = [\vec{a}_1, \dotsc, \vec{a}_n] \in
	\mat_{m,n}(\mathbb{F})$ på søjleform, og lad $A \sim H$ på RREF, således at
	$H$ har $k$ søjler med pivot.
	
	Hvor $\{\vec{a}_1, \dotsc, \vec{a}_k\}$ omnummereres til at indeholde
	søjlerne med pivoter, som er en basis for $\text{Sø}(A)$.
\end{saetning}

\begin{bevis}
	$A$ og $H$ har samme løsningsmængde da vi kan lave en lineær kombination af
	$H$
	\[
		c_1 \vec{h}_1 + \dotso + c_n \vec{h}_n = 0
	\]
	hvor $c_1,\dotsc,c_n$ findes hvis og kun hvis $A \vec{c} = 0 \Lra H \vec{c}
	= 0$.

	Da $H$ er på RREF må
	\[
		\vec{h}_i = \vec{e}_i = \begin{bmatrix}
			0 \\
			\vdots \\
			1 \\
			\vdots \\
			0
		\end{bmatrix}
	\]
	for $i=1,\dotsc,k$ og der gælder at de $k$ søjler i $H$ er uafhængige og 
	de $n-k$ andre søjler kan beskrives som en lineær kombinationer af de $k$
	søjler.
	
	Dermed må $\{\vec{a}_1, \dotsc, \vec{a}_k\}$ være en basis for
	$\text{Sø}(A)$ på grund af ækvivalencen mellem løsningsmængderne til $A$ og
	$H$.
\end{bevis}

\[
	\begin{bmatrix}
		1 & 	   & 0 & \vline \\
		  & \ddots &   & ? \\
		0 &        & 1 & \\
		\hline 
		  &   0    &   & 0
	\end{bmatrix}
\]

%	Lad $j_1 < \dotsb < j_k$ være numrene på de søjler i $H$, som inderholder
%	pivot.

