\newpage
\chapter{Vektorrum og underrum}
([P] 2.1, 3.2)

\section*{Disposition}
\begin{enumerate}
	\item Vektorrum
	\item Underrum
	\item Udspændende Mængde
	\item Basis
	\item Løsninger til homogent ligningssystem
	\item Lineær (U)afhængighed
\end{enumerate}

\section{Vektorrum}
\begin{definition}{2.1.1}
Et vektorrum er en mængde $V$ udstyret med addition og skalarmultiplikation
således at følgende egenskaber er overholdt:

	\begin{description}
		\item[A1] $\forall x,y \in V \colon x+y=y+x$
		\item[A2] $\forall x,y,z \in V \colon (x+y)+z=x+(y+z)$
		\item[A3] $\exists 0 \in V \colon x+0=x \, \forall x \in V$
		\item[A4] $\forall x \in V, \, \exists -x \in V \colon x+(-x)=0$
		\item[S1] $\forall \alpha \in \mathbb{F}, \, x,y \in V \colon
			\alpha (x+y)=\alpha x+ \alpha y$
		\item[S2] $\forall \alpha, \beta \in \mathbb{F}, \, x \in V \colon
			(\alpha + \beta)x=\alpha x + \beta x$
		\item[S2] $\forall \alpha, \beta \in \mathbb{F}, \, x \in V \colon
			(\alpha\beta)x=\alpha (\beta x)$
		\item[S4] $\exists 1 \in \mathbb{F}, \, \forall x \in V \colon 1 \cdot x=x$
	\end{description}
\end{definition}


\begin{saetning}{2.1.2}
Et vektorrum har nogle elementære egenskaber
\begin{enumerate}
		\renewcommand{\theenumi}{\Roman{enumi}}
		\renewcommand{\labelenumi}{\textbf{(\theenumi)}}
	\item 0 er det entydige neutrale element i $V$
	\item $0\cdot x=0 \, \forall x \in V$
	\item Hvis $x,y\in V$ er således at $x+y=0$, så er $y=-x$
	\item $(-1)x = -x \, \forall x\in V$
\end{enumerate}
\end{saetning}


\section{Underrum}
\hypertarget{def:underrum}{}
En delmængde $S$ af et $\mathbb{F}$-vektorrum $V$ kaldes et underrum, hvis de
har følgende egenskaber:
\begin{description}
	\item[C0] $S \not= \emptyset$
	\item[C1] $\forall x \in S, \, \alpha \in \mathbb{F} \colon \alpha x \in S$
	\item[C2] $\forall x,y \in S \colon x + y \in S$
\end{description}

\section{Udspændende Mængde}
\begin{definition}{2.2.1}
	Lad $V$ være et $\mathbb{F}$-vektorrum og lad $v_1,\dots,v_n \in V$. 

	En vektor givet ved $\alpha_1 v_1+\dots+\alpha_n v_n \in V$ hvor
	$\alpha_1,\dots,\alpha_n \in \mathbb{F}$ er en lineær kombination af
	$v_1,\dots,v_n$. 
	
	Mængden af alle lineære kombinationer af $v_1,\dots,v_n$ kaldes spannet af
$v_1,\dots,v_n$, skrevet $\spn(v_1,\dots,v_n)$.
\end{definition}


$Span(v_1,\dots,v_n)$ er et underrum af $V$, idet spannet overholder C0 - C2.
Se eventuelt beviset side 34 [duPlessis].

\section{Basis}
\begin{definition}{2.2.11}
	Lad $V$ være et $\mathbb{F}$-vektorrum, og lad $v_1, \dots, v_n \in V$. Da
	er $v_1, \dots, v_n$ en basis for $V$ hvis:
	\begin{enumerate}
		\item $v_1, \dots, v_n$ er lineært uafhængige.
		\item $v_1, \dots, v_n$ udspænder (spanner) $V$.
	\end{enumerate}
\end{definition}


\section{Løsninger til det homogene ligningssystem}
\hypertarget{1.2.13}{}
\begin{korollar}{1.2.13}
	Lad $A \in Mat_{m,n}(\mathbb{F})$.
	\begin{enumerate}
		\item Det homogene ligningssystem $Ax = 0$ har altid en løsning, 
			$x = 0$.
		\item Hvis $n > m$ så har $Ax = 0$ flere løsninger.
	\end{enumerate}
\end{korollar}
For eksempel har nedenstående $Mat_{2,3}(\mathbb{F})$ uendelige mange 
løsninger, hvis det gælder at den er lig $0$.
\[ 
	\begin{bmatrix}
		1 & 0 & 2 \\
		0 & 1 & 3
	\end{bmatrix}
	\begin{bmatrix}
		x_1 \\ 
		x_2 \\
		x_3 
	\end{bmatrix}
	=
	\begin{bmatrix}
		0 \\ 
		0 
	\end{bmatrix}
\] 
\begin{bevis}
	\begin{enumerate}
		\item Da $A0 = 0$ er $0$ en løsning til ligningssystemet.
		\item Lad $A \sim H$ på RREF, så $[A\,|\,0] = [H\,|\,0]$. $H$ har $m$
			rækker, og dermed højst $m$ pivot'er, og da $n > m$ må der altså
			være mindst $n-m$ søjler uden pivot. 
			
			Så ligningssystemet må have
			uendeligt mange løsninger hvis $\mathbb{F}$ har uendeligt mange
			elementer, eller $q^{n-m}$ løsninger, hvis $\mathbb{F}$ har $p <
			\infty$ elementer.
	\end{enumerate}
\end{bevis}


\section{Lineær (U)afhængighed}
\begin{definition}{2.2.6}
	Et sæt $v_1, \dots, v_n$ af vektorer er lineært uafhængige såfremt der
	gælder at en lineær kombination af vektorerne 
	\[c_1v_1 + \dots + c_nv_n = 0\]
	kun har den trivielle løsning, altså at $c_i = 0$ for alle $i = 1,\dotsc,
	n$. Hvis der derimod eksisterer et $c_i \not= 0$, så samme ligning stadig
	er opfyldt, så er vektorerne istedet lineært afhængige, da en kan skrives
	som en lineær kombination af de andre.
\end{definition}


\begin{saetning}{2.2.9}
	Lad $V$ være et $\mathbb{F}$-vektorrum, lad $v_1, \dots, v_n \in V$, og lad
	$S = \spn(v_1, \dots, v_n)$. Et element $v \in S$ kan udtrykkes 
	\textit{entydigt} som en lineær kombination af $v_1, \dots, v_n \Lra v_1, 
	\dots, v_n$ er lineært uafhængige.
\end{saetning}

\begin{bevis}
Da $v \in S$ ved vi per definition af spannet, at vi kan skrive $v$ som en 
lineær kombination.

\[
	v = a_1 v_1 + \dots + a_n v_n \text{ med } a_1, \dots a_n \in \mathbb{F}
\]

Hvis der samtidig gælder at vi kan skrive v som:

\[
	v = b_1 v_1 + \dots + b_n v_n \text{ med } b_1, \dots b_n \in \mathbb{F}
\]

så er 

\[
	0 = v - v = (a_1-b_1)v_1 + \dots + (a_n-b_n)v_n
\]

Hvis $v_1, \dots, v_n$ er lineært \textbf{uafhængige} så må:

\[
	(a_1-b_1) = 0, \dots, (a_n-b_n) = 0 \Lra a_1 = b_1, \dots, a_n = b_n
\]

Der findes dermed kun 1 måde at udtrykke $v$ som en lineær kombination af 
$v_1, \dots, v_n$

Hvis $v_1, \dots, v_n$ er lineært \textbf{afhængige}, må der findes $c_1, 
\dots, c_n \in \mathbb{F}$, hvor ikke alle $c_i$ er 0, og hvor $c_1 v_1 + \dots
c_n v_n = 0$, hvilket giver os:

\[
	v = v + 0 = (a_1 v_1 + \dots + a_n v_n) + (c_1 v_1 + \dots c_n v_n)
	= (a_1 + c_1)v_1 + \dots + (a_n + c_n)v_n
\]

Altså er $(a_1 + c_1)v_1 + \dots + (a_n + c_n)v_n$ et andet udtryk for en
lineær kombination af $v$, da mindst 1 $c$ ikke er 0 og ligningen derfor ikke
er lig den øverste.
\end{bevis}


\hypertarget{2.2.15}{}
\begin{saetning}{2.2.15}
	Lad $V = \spn(v_1, \dots\ v_n)$ være et $\mathbb{F}$-vektorrum. Lad 
	$u_1, \dots, u_m \in V$, hvor $m > n$. Så er $u_1, \dots, u_m$ afhængige.
\end{saetning}

\begin{bevis}
	Vi kan opstille hver af $u_i$'erne som en lineær kombination af $V$ for 
	$i = 1, \dots, m$.
	\[
		u_i = a_{1i}v_1 + \dots + a_{ni}v_n\text{, hvor }a_{1i}, \dots, a_{ni} 
		\in \mathbb{F}
	\]
	Samtidig kan vi opstille en lineær kombination af alle $u_i'er$ og omskrive
	denne til at være en sum af de ovenstående lineære kombinationer.
	\begin{align*}	
		0 = U &= x_1u_1 + \dots + x_nu_n \\
		&= \sum_{i=1}^mx_i(\sum_{j=1}^na_{ji}v_j) \\
		&= \sum_{j=1}^n(\sum_{i=1}^ma_{ji}x_i)v_j
	\end{align*}
	Vi ved at $v_1, \dots, v_n$ ikke er 0, da disse er baser. Derfor må 
	$\sum_{j=1}^n(\sum_{i=1}^ma_{ji}x_i) = 0$. Dette kan opstilles som et 
	ligningssystem der har $n$ ligninger og $m$ ubekendte:
	\begin{align*}
		x_1a_{11} + &\dots + x_ma_{1m} \\
					&\ \ \vdots \\
		x_1a_{n1} + &\dots + x_ma_{nm}
	\end{align*}
	Per \hyperlink{1.2.13}{korollar 1.2.13}, ved vi at et sådan ligningssystem
	må have en ikke triviel løsning, og per definitionen af lineær afhængighed,
	ved vi da at vektorerne $u_1, \dots, u_m$ må være lineært afhængige.
\end{bevis}


\begin{definition}{3.1.1}
	Lad $V$ være et $\mathbb{F}$-vektorrum.
	\begin{enumerate}
		\item Hvis $V = \{0\}$, så har $V$ dimension 0.
		\item Hvis $V$ har en basis bestående af $n$ vektorer, så har $V$ 
			dimension $n$.
		\item Hvis $V$ ikke har en endelig basis, så har det 
			\textit{uendelig dimension}.
	\end{enumerate}
	Et $\mathbb{F}$-vektorrum er \textit{endeligt frembragt}, hvis der findes 
	$v_1, \dotsc, v_n \in V$ så $V = \spn(v_1, \dotsc, v_n)$. Et endelig 
	frembragt vektorrum har en endelig dimension, da den udspændende mængde kan
	udtyndes til en basis.
\end{definition}

\hypertarget{3.1.4}{}
\begin{saetning}{3.1.4}
	Lad $\mathit{V}$ være et $\mathbb{F}$-vektorrum af $\dim\mathit{V} = n$,
	for n > 0.

	\begin{enumerate}
		\itemsep -0.1em
		\item Enhver mængde af $n$ uafhængige vektorer fra $\mathit{V}$ 
			udspænder $\mathit{V}$, og er derfor en basis.
		\item Enhver mængde af $n$ vektorer, som udspænder $\mathit{V}$, består
			af uafhængige vektorer, og er derfor en basis.
	\end{enumerate}
\end{saetning}

\begin{bevis}
	\begin{enumerate}
		\item Antag at $v_1, \ldots, v_n \in \mathit{V}$ er uafhængige, og lad
			$v \in \mathit{V}$ være et vilkårligt element i $\mathit{V}$.
			Ifølge 2.2.15 er de $n + 1$ vektorer $v, v_1, \ldots, v_n$
			afhængige. Altså må der findes en
			ikke-triviel løsning til
			\[cv + c_1v_1 + \cdots + c_nv_n = 0\]
			Hvor $c \ne 0$, da der ellers ville være en ikke-triviel løsning
			til $c_1v_1 + \cdots + c_nv_n = 0$.
			Derfor kan vi skrive v som
			\[v = (-c_1c^{-1})v_1 + \cdots + (-c_nc^{-1})v_n \in \spn(v_1,
			\ldots, v_n)\]
			Så er $\mathit{V} = \spn(v_1, \ldots, v_n)$.
		\item Antag at $v_1, \ldots, v_n$ udspænder $\mathit{V}$ og at
			vektorerne er afhængige.  I så fald, kan en af dem skrives som en
			lineær kombination af de andre: $v_n = c_1v_1 + \ldots +
			c_{n-1}v_{n-1}$, hvilket vil sige at $v_1, \ldots, v_{n-1}$ må
			udspænde $\textit{V}$.  Samtidig må $u_1, \ldots, u_n$ være en
			basis for $\mathit{V}$, da den er udspændt af $n$ vektorer. Per
			\hyperlink{2.2.15}{sætning 2.2.15}, må $u_1, \ldots, u_n$ være afhængige, hvilket
			er en modstrid til definitionen på en basis. Så $v_1, \ldots, v_n$
			kan ikke være afhængige.
	\end{enumerate}
\end{bevis}


\section{Backlog}
\begin{itemize}
	\item Kig evt. på lemma 3.1.6
	\item Overvej beviser fra 3.2
\end{itemize}
