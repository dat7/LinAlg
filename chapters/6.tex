\newpage
\chapter{Determinanter}
([P] 8.1, 8.2)

\section{Definition og Egenskaber}
\begin{definition}{8.1.1}
	For $A\in\mat_{n,n}(\mathbb{F})$ med $n \ge 2$ er den $(i,j)$'te
	\emph{minor} $M(A)_{ij} \in \mat_{n-1,n-1}(\mathbb{F})$ dannet ved at
	sløjfe den $i$'te søjle og $j$'te række.

	\[
		A_{ij} = \begin{bmatrix}
			a_{1,1} & \dotso & \tikz[remember picture,overlay] \node (t1)[anchor=base] {$a_{1,i}$}; & \dotso & a_{1,n}\\
			\vdots  &        & \vdots  &        & \vdots\\
			\tikz[remember picture, overlay] \node (t3){$a_{j,1}$}; & \dotso & a_{j,i} & \dotso & \tikz[remember picture, overlay] \node (t4) {$a_{j,n}$};\\
			\vdots  &        & \vdots  &        & \vdots\\
			a_{n,1} & \dotso & \tikz[remember picture,overlay] \node (t2) {$a_{n,i}$}; & \dotso & a_{n,n}
		\end{bmatrix}
		\begin{tikzpicture}[overlay,remember picture]
			\draw ([yshift=-1mm]t1.north) to ([yshift=2mm]t2.south);
			\draw ([yshift=.5mm, xshift=1mm]t3.west) to ([yshift=0.5mm, xshift=-1mm]t4.east);
		\end{tikzpicture}
		=
		\begin{bmatrix}
			a_{1,1} & \dotso & a_{1,i-1} & a_{1, i+1} & \dotso & a_{1,n}\\
			\vdots  &        & \vdots    & \vdots     &        & \vdots\\
			a_{j-1,1} & \dotso & a_{j-1,i-1} & a_{j-1, i+1} & \dotso & a_{j-1,n}\\
			a_{j+1,1} & \dotso & a_{j+1,i-1} & a_{j+1, i+1} & \dotso & a_{j+1,n}\\
			\vdots  &        & \vdots    & \vdots     &        & \vdots\\
			a_{n,1} & \dotso & a_{n,i-1} & a_{n, i+1} & \dotso & a_{n,n}
		\end{bmatrix}
	\]
\end{definition}

\begin{definition}{8.1.2}
	For $B \in \mat_{1,1}(\mathbb{F})$ er $\det{B} = b_{1,1}$.

	\noindent
	For $A \in \mat_{n,n}(\mathbb{F})$ og $n > 1$ antager vi at det er vist for
	en $(n-1)\times(n-1)$-matrix.
	\begin{enumerate}
		\item For $1 \le i,j\le n$ er den $(i,j)$'te \emph{cofaktor} $A_{ij}$
			af $A$ givet som
			\[
				A_{ij}= (-1)^{i+j}\det(M(A)_{ij})
			\]
		\item Determinanten af $A$ er givet som
			\[
				\det A = a_{11}A_{11}+\dotsb+a_{1n}A_{1n}
			\]
	\end{enumerate}
	Determinanten \emph{udvikles} altså efter første række.
\end{definition}

\begin{bemaerk}
Der er en række basale regler for determinanter:
\begin{itemize}
	\item \emph{8.1.3}: Determinanten kan udvikles efter første \emph{søjle}:
		$\det(A) = a_{11}A_{11} + \dotsb + a_{n1}A_{n1}$
	\item \emph{8.1.4}: Hvis $B$ er $A$ med de første to rækker byttet om, så
		er $\det A = -\det B$.
	\item \emph{8.1.5}: Hvis $B$ er $A$ med to vilkårlige rækker byttet gælder
		$\det A = -\det B$.
	\item \emph{8.1.6}: Determinanten kan udvikles efter $k$'te række:
		$\det(A) = a_{k1}A_{k1}+\dotsb+a_{kn}A_{kn}$.
	\item \emph{8.1.8}: Hvis $B$ er $A$ med to søjler byttet, so er $\det(A)
		= -\det(B)$.
	\item \emph{8.1.9}: For $A\in\mat_{n,n}(\mathbb{F})$ og $n > 1$, hvis $A$
		har to ens rækker/søjler gælder det at $\det(A) = 0$.
	\item \emph{8.1.10/11 (a)}: Hvis $A$ har en nul-række/søjle er $\det(A) =
		0$.
	\item \emph{8.1.10/11 (b)}: Hvis $B$ er $A$ med $k$'te række/søjle gange
		med $r$, så gælder $\det(B) = r\det(A)$
	\item \emph{8.1.10/11 (c)}: Hvis $B$ er $A$ med $s$ gange $i$'te
		række/søjle lagt til $j$'te række, $i \ne j$, så er $\det(A) =
		\det(B)$.
	\item \emph{8.1.12}: Hvis $A$ er triangulær, så er $\det(A)$ produktet af
		$A$'s diagonalindgange.
	\item \emph{8.1.13}: Hvis $E$ er elementærmatrix gælder $\det(EA) = \det(E)
		\det(A)$. Deraf følger $\det(AE) = \det(E)\det(A)$.
\end{itemize}
\end{bemaerk}

\section{Transponeret Matrix}
\begin{lemma}{8.1.7}
	Lad $A \in Mat_{n,n}(\mathbb{F})$. Der gælder $\det(A^T) = \det(A)$.
\end{lemma}

\begin{bevis}
	Vi beviser lemmaet ved induktion. Vores basistilfælde kan være for matricen 
	hvor $n = 2$.
	\begin{align*}
		\det(A) &= \det\begin{pmatrix}
			a & b \\
			c & d
		\end{pmatrix} = ad - bc\\
		\det(A^T) &= \det\begin{pmatrix}
			a & c \\
			b & d
		\end{pmatrix} = ad - cb
	\end{align*}
	I ovenstående må der gælde lighedstegn, da den kommutative lov er gældende.
	
	\noindent Herefter opstiller vi induktions hypotesen at $\det(A^T) = 
	\det(A)$ når $A$ er en $(n-1) \times (n-1)$ matrix.
	
	\noindent Vi vil herefter via induktion vise at dette også gør sig gældende
	for en $n \times n$ matrix.
	\[
		\det(A^T) = \sum_{j=1}^{n}(-1)^{1+j}a_{1j}\det(M(A^T)_{j1})
	\]
	Vi kan da skrive minoren som $M(A^T)_{j1} = (M(A)_{1j})^T$. Da må det per 
	induktionshypotesen gælde:
	\[
		\det(M(A^T)_{j1}) = \det(M(A)_{1j}^T) = \det(M(A)_{1j})
	\]
	Da minoren netop er en $(n-1) \times (n-1)$ matrix. Derfor må:
	\[
		\det(A^T) = \sum_{j=1}^{n}(-1)^{1+j}a_{1j}\det(M(A)_{1j}) = \det(A)
	\]
\end{bevis}


\section{Singulære Matricer og Determinanter}

%
% Sætning 8.1.15 ([L] 2.2.2) side 144
%

\begin{saetning}{8.1.15}
	$A \in \mat_{n,n}(\mathbb{F})$ er singulær $\Lra$ $\det(A)=0$.
\end{saetning}

\begin{bevis}
	$A \sim H$, hvor $H$ er på RREF.
	\[
		A=E_k E_{k-1} \cdots E_1 H
	\]
	hvilket giver en determinant
	\begin{align*}
		\det(A) =& \det(E_k E_{k-1} \cdots E_1 H) \\
			=& \det(E_k)\det(E_{k-1} \cdots E_1 H) \\
			=& \det(E_k)\det(E_{k-1}) \cdots \det(E_1)\det(H)
	\end{align*}

	\begin{itemize}
		\item ($\Ra$) $H$ må have en nulrække, hvilket giver $\det(H) = 0 \Ra 
			\det(A) = 0$.
		\item ($\La$) $H=I$; $H$ er ikke singulær hvilket betyder at $\det(A)
			\ne 0$.
	\end{itemize}
\end{bevis}


\section{Cramers Regel}

\begin{definition}{8.2.6}
	Lad $A \in \mat_{n,n}(\mathbb{F})$. Den adjungerede matrix er da
	%\[
	%	\adj A = 
	%	\begin{bmatrix}
	%		A_{11} & A_{21} & \cdots & A_{n1}\\
	%		A_{12} & \tikz[remember picture, overlay] \node (d1) {}; &  & \vdots \\
	%		\vdots &  & \tikz[remember picture, overlay] \node (d2) {}; & \vdots \\
	%		A_{1n} & \cdots & \cdots & A_{nn}
	%	\end{bmatrix}
	%\]
	%\begin{tikzpicture}[remember picture, overlay]
	%	\draw[dotted,draw] (d1.north west) -- (d2.south east);
	%\end{tikzpicture}
	\[
		\adj A = 
		\begin{bmatrix}
			A_{11} & A_{21} & \cdots & A_{n1}\\
			A_{12} & \ddots &  & \vdots \\
			\vdots &  & \ddots & \vdots \\
			A_{1n} & \cdots & \cdots & A_{nn}
		\end{bmatrix}
	\]
	svarende til den transponerede kofaktormatrice.
\end{definition}

\begin{korollar}{8.2.9}
	Lad $A \in \mat_{n,n}(\mathbb{F})$ være invertibel, og lad $\vec{b} \in 
	\mathbb{F}^n$.
	Lad $A_i$ være matricen, der fås ved at erstatte den $i$'te søjle i $A$ med
	$\vec{b}$. Den entydige løsning $\hat{x}$ til $A\vec{x} = \vec{b}$ er da 
	givet ved:
	\[
		\hat{x}_i = \frac{\det(A_i)}{\det(A)}, i = 1, \cdots, n
	\]
\end{korollar}

\begin{bevis}
	Hvis vi ombytter lidt på ligningssystemet $A\vec{x} = \vec{b}$, får vi at
	$\vec{x} = A^{-1}\vec{b}$. Dette kan vi ved hjælp at Korrolar 8.2.8 
	omskrive:
	\[
		\vec{\hat{x}} = A^{-1}\vec{b} = \frac{1}{\det(A)}adj(A)\vec{b}
	\]
	hvor $\hat{\vec{x}} = \begin{bmatrix}\hat{x}_1\\ \vdots \\ \hat{x}_n 
	\end{bmatrix}$, og derfor må det gælde at for $i = 1, \cdots, n$:
	\begin{align*}
		\hat{\vec{x}}_i	&= \frac{1}{\det(A)}(\text{adj }A)_{i:}\vec{b} 
							& \text{(den i'te række)}\\
						&= \frac{1}{\det(A)}(A_{1i}b_1 + \cdots + A_{ni}b_n)\\
						&= \frac{1}{\det(A)}\det(A_i)\\
						&= \frac{\det(A_i)}{\det(A)}
	\end{align*}
\end{bevis}

