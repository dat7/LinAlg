\begin{saetning}{3.2.6}
	Lad $A = [\vec{a}_1, \dotsc, \vec{a}_n] \in
	\mat_{m,n}(\mathbb{F})$ på søjleform, og lad $A \sim H$ på RREF, således at
	$H$ har $k$ søjler med pivot.
	
	Hvor $\{\vec{a}_1, \dotsc, \vec{a}_k\}$ omnummereres til at indeholde
	søjlerne med pivoter, som er en basis for $\text{Sø}(A)$.
\end{saetning}

\begin{bevis}
	$A$ og $H$ har samme løsningsmængde da vi kan lave en lineær kombination af
	$H$
	\[
		c_1 \vec{h}_1 + \dotso + c_n \vec{h}_n = 0
	\]
	hvor $c_1,\dotsc,c_n$ findes hvis og kun hvis $A \vec{c} = 0 \Lra H \vec{c}
	= 0$.

	Da $H$ er på RREF må
	\[
		\vec{h}_i = \vec{e}_i = \begin{bmatrix}
			0 \\
			\vdots \\
			1 \\
			\vdots \\
			0
		\end{bmatrix}
	\]
	for $i=1,\dotsc,k$ og der gælder at de $k$ søjler i $H$ er uafhængige og 
	de $n-k$ andre søjler kan beskrives som en lineær kombinationer af de $k$
	søjler.
	
	Dermed må $\{\vec{a}_1, \dotsc, \vec{a}_k\}$ være en basis for
	$\text{Sø}(A)$ på grund af ækvivalencen mellem løsningsmængderne til $A$ og
	$H$.
\end{bevis}

\[
	\begin{bmatrix}
		1 & 	   & 0 & \vline \\
		  & \ddots &   & ? \\
		0 &        & 1 & \\
		\hline 
		  &   0    &   & 0
	\end{bmatrix}
\]

%	Lad $j_1 < \dotsb < j_k$ være numrene på de søjler i $H$, som inderholder
%	pivot.
