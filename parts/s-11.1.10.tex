%
% Sætning 11.1.10 (Spektralsætning for hermite'ske matricer, [L] 6.4.4) side 197
%

\begin{saetning}{11.1.10}
	Lad $A \in \mat_{n,n}(\mathbb{C})$ være hermite'sk. Så kan $A$ 
	diagonaliseres unitært, det vil sige der findes en unitær matrix 
	$U \in \mat_{n,n}(\mathbb{C})$ så
	\[
		U^{-1}AU=U^HAU
	\]
	er diagonal, med reelle diagonalindgange.
\end{saetning}

\begin{bevis}
	Ifølge Schurs sætning findes der en unitær matrix $U \in \mat(\mathbb{C})$
	så $U^HAU = T$ er en øvretrekantsmatrix. $T$ er hermite'sk da
	\begin{equation}\label{eqn:11.1.10-1}
		T^H = (U^HAU)^H = U^HA^H(U^H)^H = U^HAU = T
	\end{equation}
	Da $T$ er øvretriangulær må $T^H$ være nedretriangulær
	\[
		T = \begin{bmatrix}
			t_{11} & \dots 	& t_{1n} \\
				   & \ddots & \vdots \\
			   0   & 		& t_{nn} \\
		\end{bmatrix} \; , \; T^H = \begin{bmatrix}
			\overline{t_{11}} & 	   &    0 \\
				\vdots		  & \ddots &  \\
			\overline{t_{1n}} & \dots  & \overline{t_{nn}} \\
		\end{bmatrix}
	\]
	men (\ref{eqn:11.1.10-1}) siger også at $T^H = T$, så for $i \not= j$ er
	$t_{ij} = 0$ og $t_{ii} = \overline{t_{ii}}$ for $i = 1,\dotsc,n$, hvilket
	betyder at $T$ er diagonal med reelle indgange.
\end{bevis}
