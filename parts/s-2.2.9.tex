\begin{saetning}{2.2.9}
	Lad $V$ være et $\mathbb{F}$-vektorrum, lad $v_1, \dots, v_n \in V$, og lad
	$S = \spn(v_1, \dots, v_n)$. Et element $v \in S$ kan udtrykkes 
	\textit{entydigt} som en lineær kombination af $v_1, \dots, v_n \Lra v_1, 
	\dots, v_n$ er lineært uafhængige.
\end{saetning}

\begin{bevis}
Da $v \in S$ ved vi per definition af spannet, at vi kan skrive $v$ som en 
lineær kombination.

\[
	v = a_1 v_1 + \dots + a_n v_n \text{ med } a_1, \dots a_n \in \mathbb{F}
\]

Hvis der samtidig gælder at vi kan skrive v som:

\[
	v = b_1 v_1 + \dots + b_n v_n \text{ med } b_1, \dots b_n \in \mathbb{F}
\]

så er 

\[
	0 = v - v = (a_1-b_1)v_1 + \dots + (a_n-b_n)v_n
\]

Hvis $v_1, \dots, v_n$ er lineært \textbf{uafhængige} så må:

\[
	(a_1-b_1) = 0, \dots, (a_n-b_n) = 0 \Lra a_1 = b_1, \dots, a_n = b_n
\]

Der findes dermed kun 1 måde at udtrykke $v$ som en lineær kombination af 
$v_1, \dots, v_n$

Hvis $v_1, \dots, v_n$ er lineært \textbf{afhængige}, må der findes $c_1, 
\dots, c_n \in \mathbb{F}$, hvor ikke alle $c_i$ er 0, og hvor $c_1 v_1 + \dots
c_n v_n = 0$, hvilket giver os:

\[
	v = v + 0 = (a_1 v_1 + \dots + a_n v_n) + (c_1 v_1 + \dots c_n v_n)
	= (a_1 + c_1)v_1 + \dots + (a_n + c_n)v_n
\]

Altså er $(a_1 + c_1)v_1 + \dots + (a_n + c_n)v_n$ et andet udtryk for en
lineær kombination af $v$, da mindst 1 $c$ ikke er 0 og ligningen derfor ikke
er lig den øverste.
\end{bevis}
