\begin{saetning}{12.2.2 (Putzers Algoritme)}
	Lad $A\in \mat_{n,n}(\mathbb{C})$ og $\lambda_1, \dotsc, \lambda_n$ være
	$A$'s egenværdier, talt med multiplicitet.

	Vi lader $P_k$ være
	\[
		P_k = \begin{cases}
			I & \text{ hvis } k= 0\\
			\prod_{j=1}^k(A-\lambda_jI) & \text{ for } k = 1, \dotsc, n
		\end{cases}
	\]
	og kan nu definere $Q(t)$ som
	\[
		Q(t) = \sum_{k=0}^{n-1}r_{k+1}(t)P_k
	\]
	hvor $r_k$ er
	\[
		r_k(t) =
		\begin{cases}
			e^{\lambda_1t} & \text{ hvis } k = 1\\
			e^{\lambda_kt}\int\limits_0^te^{-\lambda_ks}r_{k-1}(s) \:ds & \text{ for }
			k = 2, \dotsc, n
		\end{cases}
	\]

	Så gælder at $Q(0) = I$, og $Q'(t) = AQ(t) = Q(t)A$.
\end{saetning}

\begin{bevis}
	Først kan vi sige at $r_1(0) = 1$, og $r_k(0) = 0$ for $k = 2, \dotsc, n$,
	så $Q(0) = P_0 = I$.

	Vi kan se at $A$ kommuterer med $(A - \lambda_iI$ for $i = 1, \dotsc, n$,
	derfor kommuterer den også med $P_0, \dotsc, P_{n-1}$, og så med $Q(t)$.
	Altså gælder det at $AQ(t) = Q(t)A$.

	For $k > 1$ gælder ddet at
	\begin{align*}
		r'_k(t) &= (e^{\lambda_kt}\int\limits_0^te^{-\lambda_ks}r_{k-1}(s) \:ds)'\\
		&= \lambda_ke^{\lambda_kt}\int\limits_0^te^{-\lambda_ks}r_{k-1}(s)\:ds
		+ e^{\lambda_kt}e^{-\lambda_kt}r_{k-1}(t) && \text{kæde reglen}\\
		&= \lambda_kr_k(t) + r_{k-1}(t)
	\end{align*}

	Ved at definere $r_0(t) = 0$, så gælder dette også for $k = 1$.

	Vi har så at
	\begin{align*}
		Q'(t) &= \sum_{k=0}^{n-1}r'_{k+1}(t)P_k\\
		&= \sum_{k=0}^{n-1}(\lambda_{k+1}r_{k+1}(t) + r_k(t))P_k
	\end{align*}

	og derfor
	\begin{align*}
		Q'(t) - AQ(t) &= \sum_{k=0}^{n-1}(\lambda_{k+1}r_{k+1}(t) + r_k(t))P_k
		- A\sum_{k=0}^{n-1} r_{k+1}(t)P_k\\
		&= \sum_{k=0}^{n-1} (-r_{k+1}(t)(A - \lambda_{k+1}I)P_k + r_k(t)P_k)\\
		&= \sum_{k=0}^{n-1} (-r_{k+1}(t)P_{k+1}+r_k(t)P_k) \tag{$\clubsuit$}\\
		&= -r_n(t)P_n \tag{$\sharp$}\\
		&= 0
	\end{align*}

	I $(\clubsuit)$ udnytter vi at $r_0(t)P_0 = 0$, og at alle andre end 
	$-r_n(t)P_n$ går ud med hinanden.

	I $(\sharp)$ siger Cayley-Hamilton at $P_n = 0$. Beviset er fuldført.
\end{bevis}
