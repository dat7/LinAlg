\begin{lemma}{8.1.7}
	Lad $A \in Mat_{n,n}(\mathbb{F})$. Der gælder $\det(A^T) = \det(A)$.
\end{lemma}

\begin{bevis}
	Vi beviser lemmaet ved induktion. Vores basistilfælde kan være for matricen 
	hvor $n = 2$.
	\begin{align*}
		\det(A) &= \det\begin{pmatrix}
			a & b \\
			c & d
		\end{pmatrix} = ad - bc\\
		\det(A^T) &= \det\begin{pmatrix}
			a & c \\
			b & d
		\end{pmatrix} = ad - cb
	\end{align*}
	I ovenstående må der gælde lighedstegn, da den kommutative lov er gældende.
	
	\noindent Herefter opstiller vi induktions hypotesen at $\det(A^T) = 
	\det(A)$ når $A$ er en $(n-1) \times (n-1)$ matrix.
	
	\noindent Vi vil herefter via induktion vise at dette også gør sig gældende
	for en $n \times n$ matrix.
	\[
		\det(A^T) = \sum_{j=1}^{n}(-1)^{1+j}a_{1j}\det(M(A^T)_{j1})
	\]
	Vi kan da skrive minoren som $M(A^T)_{j1} = (M(A)_{1j})^T$. Da må det per 
	induktionshypotesen gælde:
	\[
		\det(M(A^T)_{j1}) = \det(M(A)_{1j}^T) = \det(M(A)_{1j})
	\]
	Da minoren netop er en $(n-1) \times (n-1)$ matrix. Derfor må:
	\[
		\det(A^T) = \sum_{j=1}^{n}(-1)^{1+j}a_{1j}\det(M(A)_{1j}) = \det(A)
	\]
\end{bevis}
