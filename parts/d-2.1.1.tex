\begin{definition}{2.1.1}
Et vektorrum er en mængde $V$ udstyret med addition og skalarmultiplikation
således at følgende egenskaber er overholdt:

	\begin{description}
		\item[A1] $\forall x,y \in V \colon x+y=y+x$
		\item[A2] $\forall x,y,z \in V \colon (x+y)+z=x+(y+z)$
		\item[A3] $\exists 0 \in V \colon x+0=x \, \forall x \in V$
		\item[A4] $\forall x \in V, \, \exists -x \in V \colon x+(-x)=0$
		\item[S1] $\forall \alpha \in \mathbb{F}, \, x,y \in V \colon
			\alpha (x+y)=\alpha x+ \alpha y$
		\item[S2] $\forall \alpha, \beta \in \mathbb{F}, \, x \in V \colon
			(\alpha + \beta)x=\alpha x + \beta x$
		\item[S2] $\forall \alpha, \beta \in \mathbb{F}, \, x \in V \colon
			(\alpha\beta)x=\alpha (\beta x)$
		\item[S4] $\exists 1 \in \mathbb{F}, \, \forall x \in V \colon 1 \cdot x=x$
	\end{description}
\end{definition}
