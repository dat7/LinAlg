\begin{korollar}{8.2.8}
	Hvis $A \in Mat_{n,n}(\mathbb{F})$ er invertibel, så gælder det at
	\[
		A^{-1} = \frac{1}{\det(A)}adj(A)
	\]
\end{korollar}
\begin{proof}
	\[
		A(\frac{1}{\det(A)}adj(A)) = \frac{1}{\det(A)}Aadj(A) = 
		\frac{1}{\det(A)}\det(A)I
	\]
	Hvor det sidste lighedstegn fås fra sætning 8.2.7.
\end{proof}
