%
% Til Kapitel 10.1 side 173
%

En differentialligning kan skrives som en funktion som afbilleder en vektor
$\vec{x}$ til dens differentierede.
\[
	\vec{x}'=\vec{f}(\vec{x})
\]
Da $\vec{f}$ er en lineærtransformation har den en matrixrepræsentation $A$ så 
systemet kan skrives som
\[
	\vec{x}'=A \vec{x}
\]

Her undersøger vi først tilfældet hvor $A$ er en diagonalmatrix
\[
	A = \text{diag}(\lambda_1, \dotsc, \lambda_n) =  \begin{bmatrix}
		\lambda_1 &        & 0 \\
		          & \ddots & \\
		0         &        & \lambda_n
	\end{bmatrix}
\]
