\begin{korollar}{1.2.13}
	Lad $A \in Mat_{m,n}(\mathbb{F})$.
	\begin{enumerate}
		\item Det homogene ligningssystem $Ax = 0$ har altid en løsning, 
			$x = 0$.
		\item Hvis $n > m$ så har $Ax = 0$ flere løsninger.
	\end{enumerate}
\end{korollar}
For eksempel har nedenstående $Mat_{2,3}(\mathbb{F})$ uendelige mange 
løsninger, hvis det gælder at den er lig $0$.
\[ 
	\begin{bmatrix}
		1 & 0 & 2 \\
		0 & 1 & 3
	\end{bmatrix}
	\begin{bmatrix}
		x_1 \\ 
		x_2 \\
		x_3 
	\end{bmatrix}
	=
	\begin{bmatrix}
		0 \\ 
		0 
	\end{bmatrix}
\] 
\begin{bevis}
	\begin{enumerate}
		\item Da $A0 = 0$ er $0$ en løsning til ligningssystemet.
		\item Lad $A \sim H$ på RREF, så $[A\,|\,0] = [H\,|\,0]$. $H$ har $m$
			rækker, og dermed højst $m$ pivot'er, og da $n > m$ må der altså
			være mindst $n-m$ søjler uden pivot. 
			
			Så ligningssystemet må have
			uendeligt mange løsninger hvis $\mathbb{F}$ har uendeligt mange
			elementer, eller $q^{n-m}$ løsninger, hvis $\mathbb{F}$ har $q <
			\infty$ elementer.
	\end{enumerate}
\end{bevis}
