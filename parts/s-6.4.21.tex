\begin{saetning}{6.4.21}
	Lad $Q \in \mat_{n,n}(\mathbb{R})$. Følgende udsagn er ækvivalente:
	\begin{enumerate}[(a)]
		\item $Q$ er ortogonal
		\item $Q^TQ = I$
		\item $Q$ er invertibel og $Q^{-1} = Q^T$
		\item $(Q\vec{x})^T(Q\vec{y}) = \vec{x}^T\vec{y}$ for alle $\vec{x}, 
			\vec{y} \in \mathbb{R}^n$
		\item $\norm{Q\vec{x}} = \norm{\vec{x}}$ for alle $\vec{x} \in 
			\mathbb{R}^n$
	\end{enumerate}
\end{saetning}

\begin{bevis}
	\begin{description}
		\item[$(a)\Ra(b)$] følger fra Lemma 6.4.18, da søjlerne da må være en 
			ortonormal mængde.
		\item[$(b)\Ra(d)$] $(Q\vec{x})^T(Q\vec{y}) = \vec{x}^TQ^TQ\vec{y} = 
			\vec{x}^T\vec{y}$ for alle $\vec{x}, \vec{y} \in \mathbb{R}^n$.
		\item[$(d)\Ra(a)$] Vi kan opskrive $Q = [\vec{q_1}, \dotsc, \vec{q_n}]$
			i søjleform. Der gælder at:
			\[
				\vec{q_i}^T\vec{q_j} = (Q\vec{e_i})^T(Q\vec{e_j}) = 
				\vec{e_i}^T\vec{e_j} = \delta_{ij}
			\]
			hvor det gælder for $\delta_{ij}$:
			\[
				\delta_{ij} = \begin{cases}1 & i = j\\0 & i \not= j\end{cases}
			\]
			hvilket per definition gør at $\{q_1, \dotsc, q_n\}$ er ortonomal.
		\item[$(b)\Ra(c)$] følger af Lemma 1.4.10, som siger at $AB = I$ så må
			$A$ og $B$ være invertible.
		\item[$(c)\Ra(b)$] følger af at være invers.
		\item[$(d)\Ra(e)$] $\norm{Q\vec{x}}^2 = (Q\vec{x})^TQ\vec{x} = 
			\vec{x}^T\vec{x} = \norm{\vec{x}}^2$.
		\item[$(e)\Ra(d)$] følger af propertition 6.3.5, hvor 
			$\vec{x}^T\vec{y} = \frac{1}{4}(\norm{x + y}^2-\norm{x-y}^2)$ og 
			$(Q\vec{x})^T(Q\vec{y})$ kan omskrives nogenlunde på samme måde, 
			således at det passer.
	\end{description}
\end{bevis}
