\begin{definition}{4.1.1}
	Lad $V, W$ være $\mathbb{F}$-vektorrum.
	En lineær transformation $L: V \rightarrow W$ er en afbildning, som 
	respekterer lineær struktur, dvs.
	\begin{enumerate}
		\item $\forall \vec{v_1}, \vec{v_2} \in V, L(\vec{v_1} + \vec{v_2}) = 
			L(\vec{v_1}) + L(\vec{v_2})$.
		\item $\forall \alpha \in \mathbb{F}, \vec{v} \in V, L(\alpha\vec{v}) =
			\alpha L(\vec{v})$.
	\end{enumerate}
	En \textit{lineær transformation} kaldes også for en \textit{lineær
	afbildning}, og en \textit{lineær operator} Hvis de 2 rum $V$ og $W$ er
	ens.
	Følgende egenskaber gør sig gældende for en lineær transformation:
	\begin{enumerate}
		\item $L(0_{\mathcal{V}}) = 0_{\mathcal{W}}$.
		\item $L$ respekterer lineære kombinationer dvs.
			\[
				L(\alpha_1\vec{v_1} + \cdots + \alpha_n\vec{v_n}) = 
				\alpha_1L(\vec{v_1}) + \cdots + \alpha_nL(\vec{v_n})
			\]
		\item $L(-\vec{v}) = -L(\vec{v}), \forall \vec{v} \in V$
	\end{enumerate}
\end{definition}
