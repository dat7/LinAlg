\begin{saetning}{11.1.9 (Schurs Sætning)}
	Lad $A\in \mat_{n,n}(\mathbb{C})$. Der findes en unitær matrix $U\in \mat_{
	n,n}(\mathbb{C})$, så $U^HAU$ er en øvretriangulær matrix.
\end{saetning}

\begin{bevis}
	\setlength{\parindent}{0cm}
	Beviset foregår ved induktion over $n$. I basen $n=1$ er det trivielt
	opfyldt, da alle matricer $\in \mat_{1,1}$ er øvre triangulære.
	
	\noindent
	Antag nu at det gælder for en matrix $B \in \mat_{n-1,n-1}(\mathbb{C})$.
	
	Lad $A\in \mat_{n,n}(\mathbb{B})$ med en egenværdi $\lambda_1 \in
	\mathbb{C}$, og en tilhørende egenvektor $\vec{v}_1 \in \mathbb{C}^n$.

	Vi kan arrangere at $\norm{\vec{v}_1} = 1$. Og vi kan udvide
	$\{\vec{v}_1\}$ til en basis for $\mathbb{C}^n$, og ved at anvende
	Gram-Schmidt på denne basis, får vi en ortonormalbasis $\{\vec{v}_1, \dotsc
	,\vec{v}_n\}$ for $\mathbb{C}^n$.

	Vi lader $U_1 = [\vec{v}_1, \dotsc, \vec{v}_n] \in \mat_{n,n}(\mathbb{C})$,
	hvilket gør $U_1$ unitær.

	Vi vil nu forsøge at afgøre opbygningen af den første søjle i $U_1^HAU_1$.
	Dette kan vi gøre ved at gange med $\vec{e}_1$:
	\[
		U_1^HAU_1\vec{e}_1
		= U_1^HA\vec{v}_1
		= U_1^H\lambda_1\vec{v}_1
		= \lambda_1U_1\vec{v}_1
		= \lambda_1U_1^HU_1\vec{e}_1
		= \lambda_1\vec{e}_1
	\]

	Vi ved derfor at første i $U_1^HAU_1$ er på formen $(\lambda_1, 0,
	\dotsc,0)^T$, og kan derfor skrive $U_1^HAU_1$ som
	\[
		U_1^HAU_1 = \begin{bmatrix}
			\lambda_1 & \vec{r}\\
			\vec{0}   & A_1
		\end{bmatrix}
	\]

	Her er $\vec{r}$ en rækkevektor og $\vec{0}$ en søjlevektor, begge med
	$n-1$ indgange, og $A_1$ er defineret som minoren $M(A)_{1,1}$.
	
	Per induktionshypotesen findes der en unitær matrix $C \in \mat_{n-1,n-1}$
	der opfylder $C^HA_1C=B$, så $B$ er en øvre triangulær matrix $\in
	\mat_{n-1,n-1}(\mathbb{C})$.

	Vi definerer
	\[
		U_2 = \begin{bmatrix}
			1       & \vec{0}^T\\
			\vec{0} & C
		\end{bmatrix}
	\]
	hvor $0$ igen er en søjlevektor med $n-1$ indgange.

	$U_2$ er unitær, da de sidste $n-1$ søjler udgør en ortonormal mængde, per
	definition, og hver for sig er ortogonale på $(U_2)_{:1}$, som er lig 
	enhedsvektoren $e_1$.

	Ifølge lemma 11.1.16 er $U=U_1U_2$ ogås unitær.

	Da $U_1^HAU_1$ og $U_2$ har samme former ($n \times n$), kan vi udregne
	$U_2^H(U_1^HAU_1)U_2$ ved direkte brug af matrix-regler
	\begin{align*}
		U^HAU &= U_2^H(U_1^HAU_1)U_2\\
		&= \begin{bmatrix}
			1 & \vec{0}^T\\
			\vec{0} & C^H
		\end{bmatrix}
		\begin{bmatrix}
			\lambda_1 & \vec{r}\\
			\vec{0}   & A_1
		\end{bmatrix}
		\begin{bmatrix}
			1 & \vec{0}^T\\
			\vec{0} & C
		\end{bmatrix}\\
		&= \begin{bmatrix}
			\lambda_1 & \vec{r}\\
			\vec{0}   & C^HA_1
		\end{bmatrix}
		\begin{bmatrix}
			1 & \vec{0}^T\\
			\vec{0} & C
		\end{bmatrix}\\
		&= \begin{bmatrix}
			\lambda_1 & \vec{r}C\\
			\vec{0}   & C^HA_1C
		\end{bmatrix}\\
		&= \begin{bmatrix}
			\lambda_1 & \vec{r}C\\
			\vec{0}   & B
		\end{bmatrix}
	\end{align*}
	Som er øvre triangulær fordi $B$ er det.

	Induktionsskridtet er derfor taget, og sætningen er bevist.
\end{bevis}
