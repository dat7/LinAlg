%
% Lemma 9.2.2 ([L] 6.3.2) side 163
%

\begin{lemma}{9.2.2}
	Lad $A \in \mat_{n,n}(\mathbb{F})$. Følgende er ækvivalente
	\begin{enumerate}[(1)]
		\item Der findes en basis for $\mathbb{F}^n$ bestående af egenvektorer
			for $A$.
		\item Der findes $n$ lineært uafhængige egenvektorer for $A$.
		\item Der findes en invertibel matrix $V \in \mat_{n,n}(\mathbb{F})$ så
			$V^{-1}AV$ er en diagonalmatrix.
	\end{enumerate}
\end{lemma}

\begin{bevis}
	\begin{description}
		\item[(1) $\Ra$ (2)] Oplagt, da en basis er uafhængige.
		\item[(2) $\Ra$ (1)] Hvis man har $n$ lineært uafhængige vektorer
			danner de en basis, se \hyperlink{3.1.4}{sætning 3.1.4}.
		\item[(2) $\Ra$ (3)] Lad $\vec{v_1}, \dotsc, \vec{v_n}$ være lineært
			uafhængige egenvektorer med tilhørende $\lambda_1, \dotsc,
			\lambda_n$ for $A$, hvor $V = [\vec{v_1}, \dotsc, \vec{v_n}]$ i
			søjleform. Vi har
			\begin{align*}
				AV = A[\vec{v_1}, \dotsc, \vec{v_n}] &= [A\vec{v_1}, \dotsc,
						A\vec{v_n}] \\
					&= [\lambda_1\vec{v_1}, \dotsc, \lambda_n\vec{v_n}] \\
					&= [\vec{v_1}, \dotsc, \vec{v_n}] D \\
					&= VD
			\end{align*}
			hvor
			\[
				D = \begin{bmatrix}
					\lambda_1 & & 0 \\
					& \ddots & \\
					0 & & \lambda_n
				\end{bmatrix}
			\]
			er diagonal. Da $V$ har uafhængige søjler er den diagonal og 
			$V^{-1}AV = V^{-1}VD = D$.
		\item[(3) $\Ra$ (2)] Antag, $X \in \mat_{n,n}(\mathbb{F})$ så
			$X^{-1}AX = D$, hvor $D$ er er diagonal, som ovenover. Da $X$ er 
			invertibel er søjlerne i $X = [\vec{x}_1, \dotsc, \vec{x}_n]$
			uafhængige.
			
			Vi har $AX = X(X^{-1}AX) = XD$, dvs. $[A\vec{x_1}, \dotsc,
			A\vec{x_n}] = [\lambda_1\vec{x_1}, \dotsc, \lambda_n\vec{x_n}]$.
			Dette giver $A\vec{x_i} = \lambda_i \vec{x_i}$, hvilket betyder
			at $\vec{x}_1, \dotsc, \vec{x}_n$ er uafhængige og egenvektorer for
			$A$.
	\end{description}
\end{bevis}
