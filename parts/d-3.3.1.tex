\begin{definition}{3.3.1}
	Lad $\mathcal{V} = \{\vec{v_1}, \cdots, \vec{v_n}\}$ være en ordnet basis 
	for $V$. Lad $\vec{v} \in V$, da kan $\vec{v}$ skrives entydig som en 
	lineær kombination af $\vec{v_1}, \cdots, \vec{v_n}$:
	\[
		v = c_1v_1 + \ldots + c_nv_n
	\]
	Der er således et entydigt element $\begin{bmatrix}c_1\\ \vdots \\ c_n
	\end{bmatrix} \in \mathbb{F}^n$, som angiver koordinaterne for $\vec{v}$ i
	$\mathcal{V}$. Dette kaldes \textit{koordinatvektoren} for $\vec{v}$ mht.
	$\mathcal{V}$ og denne skrives som:
	\[
		[\vec{v}]_\mathcal{V} = \begin{bmatrix}c_1\\ \vdots \\ c_n\end{bmatrix}
	\]
	En sådan koordinatisering bevarer lineær struktur (lemma 3.3.2), hvilket 
	vil sige at:
	\begin{enumerate}
		\item $[\vec{v} + \vec{w}]_\mathcal{V} = [\vec{v}]_\mathcal{V} + 
			[\vec{w}]_\mathcal{V}$
		\item $[r\vec{v}]_\mathcal{V} = r [\vec{v}]_\mathcal{V}$
	\end{enumerate}
\end{definition}
