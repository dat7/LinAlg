\begin{lemma}{10.1.1}
	Lad $\lambda, a \in \mathbb{C}$. Ligningen
	\[
		x' = \lambda x\tag{$\Diamond$}
	\]
	har en en entydig løsning $z:\mathbb{R}\ra\mathbb{C}$
	med $z(0) = a$ givet ved $z(t) = e^{\lambda t}a$
\end{lemma}

\begin{bevis}
	\setlength{\parindent}{0cm}
	Vi ved at $\lambda$ er på formen $c + id$. Det kan vi sætte ind i
	$e^{\lambda t}$ og differentiere:

	\newcommand{\dif}{\frac{d}{dt}}
	\begin{align*}
		\dif e^{\lambda t} &= \dif e^{(c + id)t}\\
		&= \dif e^{ct + idt}\\
		&= \dif e^{ct}e^{idt}\\
		&= \dif e^{ct}(\cos(dt) + i\sin(dt))\\
		&= ce^{ct}(\cos(dt) + i\sin(dt)) + e^{ct}(-d\sin(dt) + id\cos(dt))\\
		&= ce^{ct}(\cos(dt) + i\sin(dt)) + ide^{ct}(i\sin(dt) + \cos(dt))\\
		&= (c + id)e^{ct}(\cos(dt) + i\sin(dt))\\
		&= (c + id)e^{ct}e^{idt}\\
		&= (c + id)e^{ct + idt}\\
		&= (c + id)e^{(c + id)t}\\
		&= \lambda e^{\lambda t}\\
	\end{align*}
	Altså er $z'(t) = \lambda e^{\lambda t}a = \lambda z(t)$, og $z(t)$ er
	derfor en løsning til $(\Diamond)$. Og det er klart at $z(0) =a$.

	For at vise at alle løsninger er på samme form som $z(t)$ kan vi lade
	$y:\mathbb{R} \ra \mathbb{C}$ være en løsning til $x' = \lambda x$, med
	$y(0) = a$.

	Så har vi at
	\begin{align*}
		\dif(e^{-\lambda t}y(t)) &= -\lambda e^{-\lambda t}y(t) + e^{-\lambda t}y'(t)\\
		&= e^{-\lambda t}(y'(t) - \lambda y(t))\tag{$\star$}\\
		&= 0
	\end{align*}

	Hvilket giver os at $e^{-\lambda t}y(t)$ er konstant. Som kun kan ske hvis
	$y(t)$ er invers til $e^{-\lambda t}$, nemlig på formen $e^{\lambda t}$.
	Bemærk at $(\star)$ følger af $y'(t) = \lambda y(t)$. 
	
	Vi er også givet for $t = 0$ at
	\[
		e^{-\lambda t}y(t) = e^{-\lambda 0}y(0) = y(0) = a
	\]

	Derfor 
	\[
		y(t) = e^{\lambda t}a = z(t)
	\]

	$z(t)$ er altså den entydige løsning til $(\Diamond)$ med $z(0) = a$.
\end{bevis}
