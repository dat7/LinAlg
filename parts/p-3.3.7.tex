\begin{proposition}{3.3.7}
	Lad $\mathcal{U} = {\vec{u_1}, \ldots, \vec{u_n}}$, $\mathcal{V} = {
	\vec{v_1}, \ldots, \vec{v_n}}$ 
	være 2 ordnede baser i $W \in \mathbb{F}^n$. 
	Lad $K \in Mat_{n,n}(\mathbb{F})$ være givet i søjleform ved $[[
	\vec{u_1}]_{\mathcal{V}}, \ldots, [\vec{u_n}]_{\mathcal{V}}]$.
	Følgende gør sig da gældende:
	\begin{enumerate}
		\item $K$ er invertibel
		\item $\forall \vec{w} \in W, [\vec{w}]_{\mathcal{V}} = 
			K[\vec{w}]_{\mathcal{U}}$.
		\item $K$ er den entydige matric i $Mat_{n,n}(\mathbb{F})$, således at
			(2) gør sig gældende.
	\end{enumerate}
	Sagt med andre ord
	\[
		V = \spn(\vv_1, \dotsc, \vv_n) \Lra \vv_1, \dotsc, \vv_n \in V \text{
		er uafhængige}
	\]
\end{proposition}

\begin{bevis}
	\begin{enumerate}
		\item Vi antager at $K' = K\begin{bmatrix}x_1\\ \vdots 
			\\x_n\end{bmatrix} = 0$, så vi kan skrive $K'$ som en lineær 
			kombination:

			\[
				K' = x_1[\vec{u_1}]_{\mathcal{V}} + \ldots + 
					x_n[\vec{u_n}]_{\mathcal{V}} = [x_1\vec{u_1} + \ldots + 
					x_n\vec{u_n}]_{\mathcal{V}} = 0
			\]
			
			Hvor det sidste lighedstegn kommer af at koordinatisering bevarer
			lineær struktur.
			Da $K' = 0$ må det gælde $\forall x = 0$, da $\vec{u_1}, \ldots, 
			\vec{u_n}$ er uafhængige, per definitionen af en basis, og dermed 
			ikke kan være 0. 
			Ligningssystemet har da kun løsningen $0$ per 
			\hyperlink{1.4.8}{sætning 1.4.8}, og $K$ er dermed invertibel.
		\item For $i = 1,\dots,n$ kan $\vec{u_i}$ skrives som en lineær kombination
			$\vec{u_i} = k_{1i} \vec{v_1}+\dots + k_{ni} \vec{v_n}$ hvor
			\[
				\begin{bmatrix}
					k_{1i} \\
					\vdots \\
					k_{ni}
				\end{bmatrix} = [ \vec{u_i} ]_\mathcal{V}
			\]
			er den $i$'te søjle i $K$.

			En anden vektor $\vec{w} = c_1 \vec{u_1} + \dots + c_n \vec{u_n}$ 
			\[
				[ \vec{w} ]_\mathcal{U} = \begin{bmatrix}
					c_{1} \\
					\vdots \\
					c_{n}
				\end{bmatrix}
			\]
			Skrives denne helt ud har vi
			\begin{align*}
				\vec{w} &= c_1 (k_{11} \vec{v_1} + \dots + k_{n1} \vec{v_n}) + \dots + c_n
						(k_{1n} \vec{v_1} + \dots + k_{nn} \vec{v_n}) \\
					&= (c_1 k_{11} + \dots + c_n k_{1n}) \vec{v_1} + \dots + (c_1 k_{n1} +
						\dots + c_n k_{nn}) \vec{v_n}
			\end{align*}
			hvor
			\[
				[ \vec{w} ]_\mathcal{V} = \begin{bmatrix}
					k_{11} c_1 + \dots + k_{1n} c_n \\
					\vdots \\
					k_{n1} c_1 + \dots + k_{nn} c_n
				\end{bmatrix} = K \begin{bmatrix}
					c_1 \\
					\vdots \\
					c_n
				\end{bmatrix} = K[\vec{w}]_\mathcal{U}
			\]

			Vi har hermed vist at $K$ er \emph{koordinattransformationsmatricen} til 
			$\mathcal{V}$-koordinater fra $\mathcal{U}$-koordinater.

			\[
				[ \vec{w} ]_\mathcal{V} = K_{\mathcal{V},\mathcal{U}} [ \vec{w}
				]_\mathcal{U}
			\]
		\item Vi antager at $K'$ eksisterer så
			\[
				[\vec{w}]_\mathcal{V} = K'[\vec{w}]_\mathcal{U}, \;\forall\; \vec{w} \in
				\mathbb{F}^n
			\]
			Det gælder så for $u_i, i = 1, \ldots, n$ at
			\[
				\begin{bmatrix}
					k_{1i}\\
					\vdots\\
					k_{1n}
				\end{bmatrix}
				=
				[\vec{u}_i]_\mathcal{V} = K'[\vec{u}_i]_\mathcal{U} = K'\vec{e}_i =
				\begin{bmatrix}
					k'_{1i}\\
					\vdots\\
					k'_{1n}
				\end{bmatrix}
			\]
			for $i = 1, \ldots, n$.

			Da 
			$$
			\mathcal{U} = \{\vec{u}_1, \cdots, \vec{u}_i, \cdots, \vec{u}_n\}
			\Ra \vec{u}_i = \mathcal{U}
			\begin{bmatrix}
				0\\
				\vdots\\
				1 \tikz[baseline=-.5ex] \node[coordinate] (ith){};\\
				\vdots\\
				0
			\end{bmatrix}
			\tikz[overlay]{
				\node[fill=red!25] (ith1) at (0.8,1.5) {\textit{i}'te index};
				\path[->] (ith1) edge [bend left] (ith);
			}
			= \mathcal{U}e_i
			$$
			Så $K$ og $K'$ har samme søjler, og de er ens.
	\end{enumerate}
\end{bevis}
