\begin{definition}{9.1.1}
	\begin{enumerate}
		\item Lad $V$ være et $\mathbb{F}$-vektorrum og lad $T: V \ra V$ være
			en lineær transformation.
			$\lambda \in \mathbb{F}$ er en egenværdi for $T$, hvis der findes
			$\vec{v} \in V \backslash \{0\}$, så $T(\vec{v}) = \lambda\vec{v}$. 
			
			$\vec{v}$ er da en egenvektor for $T$ associeret til $\lambda$.
		\item Lad $A \in \mat_{n,n}(\mathbb{F})$. $\lambda \in \mathbb{F}$ er 
			en egenværdi for $A$ hvis der findes $\vec{z} \in \mathbb{F}^n 
			\backslash \{0\}$, så $A\vec{z} = \lambda \vec{z}$.
			
			$\vec{z}$ er da en egenvektor for $A$ associeret til $\lambda$.
	\end{enumerate}
	Egenvektoren findes ved at finde nulrummet $N(A-\lambda I)$, eller sagt på
	en anden måde indsætte egenværdierne på $\lambda$'s plads, hvorefter 
	matricen reduceres, for at man så kan finde egenvektoren.
	Nulrummet $N(A-\lambda_i I)$ kaldes for $E_A(\lambda_i)$.
\end{definition}
